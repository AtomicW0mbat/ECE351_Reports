%%%%%%%%%%%%%%%%%%%%%%%%%%%%%
% Cameron Williams          %
% ECE 351-51                %
% Lab Number 8              %
% 24 March 2020             %
% Instructor Philip Hagen   %
%%%%%%%%%%%%%%%%%%%%%%%%%%%%%

\documentclass[12pt]{article}

% Language and font encoding
\usepackage[english]{babel}
\usepackage[utf8x]{inputenc}
\usepackage[T1]{fontenc}
\usepackage{graphicx}
\usepackage{amsmath}
\usepackage{caption}
\usepackage{float}
\usepackage{caption}
\usepackage{subcaption}
\usepackage{rotating}
\usepackage{setspace}
\usepackage{indentfirst}
\usepackage{enumitem}
\usepackage{appendix}
\usepackage[colorlinks=true, allcolors=blue]{hyperref}
\usepackage{listings}
\usepackage{gensymb}
\usepackage{amssymb}
\usepackage[final]{pdfpages}

% Sets page size and margins
\usepackage[a4paper, margin=1in]{geometry}

%Line Spacing
\setstretch{1.5}

%-------------------Begin Editing Here---------------------
%Info for Title Page
\title{Fourier Series Approximation of a Square Wave}
\author{Cameron Williams\\ECE 351-51\\Lab Report 8}
\date{24 March 2020}
\begin{document}

%Make a Title Page
\vspace{\fill}
\maketitle
\vspace{\fill}
\clearpage

\newpage
%Introduction
\section{Introduction}
    \par The objective of this lab was to use Fourier series to approximate periodic time-domain signals. For the purposes of this lab, the square wave function pictured below was used.
    \newline
    \begin{center}
    \includegraphics{include/x_t.png}
    \end{center}
    
\newpage

%Methodology
\section{Methodology}
    \par The simplified a\_k and b\_k values for the Fourier series approximation may be seen in the Equations section. I implemented them as functions in my Python script and used them to print out the first two values of a\_k, a\_0 and a\_1. I also printed out the values of b\_1, b\_2, and b\_3. These values may be seen in the Appendix. Next, I implemented a summation of the Fourier series in my Python script and plotted the it for values N = 1, N = 3, N = 15, N = 50, N = 150, and N = 1500. These plots may be seen in the Results section.
    
%Equations
\section{Equations}
    $$ a_k = 0 $$
    $$ b_k = \frac{2}{k\pi}[1 - cos(k\pi)] $$
    
\newpage
%Results
\section*{Results}
\begin{center}
    \includegraphics[scale=0.7]{include/Figure_1.png}
    \includegraphics[scale=0.7]{include/Figure_2.png}
\end{center}

\newpage
%Questions
\section*{Questions}

\begin{enumerate}
    \item Is x(t) an even or an odd function? Explain why.
\end{enumerate}
    \par The function is odd since it is not mirrored across the y-axis ($X_n = -X_{-n}$).
    
\begin{enumerate}[resume]
    \item Based on your results from Task 1, what do you expect the values of $a_2$, $a_3$, ..., $a_n$ to be? Why?
\end{enumerate}
    \par I expect all values of $a_k$ to be 0 because that's what the equation for $a_k$ simplifies to.
    
\begin{enumerate}[resume]
    \item How does the approximation of the square wave change as the value of N increases? In what way does the Fourier series struggle to approximate the square wave?
\end{enumerate}
    \par The approximation gets closer and closer to the square wave as N increases. The Fourier series struggles to approximate the square wave at the straight vertical edges.
    
\begin{enumerate}[resume]
    \item What is occurring mathematically in the Fourier series summation as the value of N increases?
\end{enumerate}
    \par As the value of N increases, each new component gets smaller and smaller, but pushes the total x(t) ever closer to resembling the waveform being approximated.
    
\begin{enumerate}[resume]
    \item Leave any feedback on the clarity/usefulness of the purpose, deliverables, and expectations for this lab.
\end{enumerate}
    \par The purpose, deliverables, and expectations for this lab were communicated clearly.

\newpage
%Appendix
\section*{Appendix}

Python output of requested a\_k and b\_k values:
\begin{verbatim}
a_0 = 0, a_1 = 0
b_1 = 1.2732395447351628, b_2 = 0.0, b_3 = 0.4244131815783876
\end{verbatim}

\end{document}