%%%%%%%%%%%%%%%%%%%%%%%%%%%%%
% Cameron Williams          %
% ECE 351-51                %
% Lab Number 3              %
% 11 February 2020          %
% Instructor Philip Hagen   %
%%%%%%%%%%%%%%%%%%%%%%%%%%%%%

\documentclass[12pt]{article}

% Language and font encoding
\usepackage[english]{babel}
\usepackage[utf8x]{inputenc}
\usepackage[T1]{fontenc}
\usepackage{graphicx}
\usepackage{amsmath}
\usepackage{caption}
\usepackage{float}
\usepackage{caption}
\usepackage{subcaption}
\usepackage{rotating}
\usepackage{setspace}
\usepackage{indentfirst}
\usepackage{enumitem}
\usepackage{appendix}
\usepackage[colorinlistoftodos]{todonotes}
\usepackage[colorlinks=true, allcolors=blue]{hyperref}
\usepackage{listings}
\usepackage{gensymb}
\usepackage{amssymb}
\usepackage[final]{pdfpages}

% Sets page size and margins
\usepackage[a4paper, margin=1in]{geometry}

%Line Spacing
\setstretch{1.5}

%-------------------Begin Editing Here---------------------
%Info for Title Page
\title{Discrete Convolution}
\author{Cameron Williams\\ECE 351-51\\Lab Report 3}
\date{11 February 2020}
\begin{document}

%Make a Title Page
\vspace{\fill}
\maketitle
\vspace{\fill}
\clearpage

\newpage
%Introduction
\section{Introduction}
    \par The objective of this lab was to create a user-defined convolution function and test it with three functions composed of the step and ramp functions created during last lab. The user-defined convolution function outputs will be compared with the outputs of the same convolutions done via the scipy.signal.convolve function.
\newpage

%Methodology
\section{Methodology}
    \par I began by importing the step\_func() and ramp\_func() functions I created in the previous lab. I later ran into an issue with my imported functions, but I will address this in the Conclusion section. Using my imported functions, I created a new set of functions to $f_1(t)$, $f_2(t)$, and $f_3(t)$. These functions may be seen below and their plots may be seen in Figure 1 of the results section.
    $$ f_1(t) = u(t-2) - u(t-9) $$
    $$ f_2(t) = e^{-t}u(t) $$
    $$ f_3(t) = r(t-2)[u(t-2) - u(t-3)] + r(4-t)[u(t-3) - u(t-4)] $$
    \par Next, I created a convolution function. I used my convolution function to find the plot the convolutions $f_1(t) * f_2(t)$, $f_2(t) * f_3(t), and f_1(t) * f_3(t)$. I used the scipy.signal.convolve function to verify the results of my convolution function. A side-by-side of the user-defined and built-in convolutions may be seen in Figure 2 of the results section.

\newpage
%Results
\section{Results}
    \begin{figure}[hbt!]
        \centering
        \includegraphics[width=\textwidth]{include/fig1.png}
        \caption{Plots of functions provided.}
    \end{figure}
    
    \begin{figure}
        \centering
        \includegraphics[width=\textwidth]{include/fig2.png}
        \caption{Plots of convolutions.}
    \end{figure}
    
    
\newpage
%Conclusion
\section{Conclusion}
    \par I completed most of the lab without great difficulty thanks to the guidance that was provided for creating the convolution function. However, I ran into a big problem when using the built-in convolution function. The function would run for a long time and then fail, giving an error stating that "volume and kernel should have the same dimensionality". I tried searching for a solution to this error, but couldn't find anything online. I eventually discovered that my step function and ramp function were implemented in a slightly different way than they were on the key provided. While the functions worked fine for the rest of the lab, this caused an issue with the built-in scipy.convolve function. A snippet of my initial code was as follows:
    \begin{verbatim}
        y = np.zeros((len(t), 1))
    \end{verbatim}
    When I changed this line to match the key (as seen below), I no longer received the dimensionality error and I was able to successfully generate the plots comparing my convolution function to the built in one.
    \begin{verbatim}
         y = np.zeros(t.shape)
    \end{verbatim}
    
\newpage
%Questions
\section*{Questions}

\begin{enumerate}
    \item Did you work alone or with classmates on this lab? If you collaborated to get to the solution, what did that process look like?
\end{enumerate}
    \par For the majority of the lab, I worked alone. However, there were a couple times that I consulted with a nearby classmate to verify that my results were reasonable.
    
\begin{enumerate}[resume]
    \item What was the most difficult part of this lab for you, and what did your problem-solving process look like?
\end{enumerate}
    \par The most difficult part of the lab was trying to solve the dimensionality error that I discussed in the Conclusion section. After searching online for a solution and not finding one, I started with debugging by commenting out sections of the code and re-running until I found the offending line. I narrowed it down to the scipy.convolve function. From there, I found that it wasn't working with my implementation of the step and ramp functions.
    
\begin{enumerate}[resume]
    \item Did you approach writing the code with analytical or graphical convolution in mind?  Why did you chose this approach?
\end{enumerate}
    \par I approached the problem initially with graphical convolution in mind because the visualization in graphical convolution is what made me understand convolution in the first place.
    
\begin{enumerate}[resume]
    \item Leave any feedback on the clarity of lab tasks, expectations, and deliverables.
\end{enumerate}
    \par There were some guidelines written in class about the expected format of the plots. It would be nice if these were made available online as I believe they differ from what is mentioned in the lab handout.
    
\end{document}