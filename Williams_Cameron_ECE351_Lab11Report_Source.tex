%%%%%%%%%%%%%%%%%%%%%%%%%%%%%
% Cameron Williams          %
% ECE 351-51                %
% Lab Number 11             %
% 21 April 2020             %
% Instructor Philip Hagen   %
%%%%%%%%%%%%%%%%%%%%%%%%%%%%%

\documentclass[12pt]{article}

% Language and font encoding
\usepackage[english]{babel}
\usepackage[utf8x]{inputenc}
\usepackage[T1]{fontenc}
\usepackage{graphicx}
\usepackage{amsmath}
\usepackage{caption}
\usepackage{float}
\usepackage{caption}
\usepackage{subcaption}
\usepackage{rotating}
\usepackage{setspace}
\usepackage{indentfirst}
\usepackage{enumitem}
\usepackage{appendix}
\usepackage[colorlinks=true, allcolors=blue]{hyperref}
\usepackage{listings}
\usepackage{gensymb}
\usepackage{amssymb}
\usepackage[final]{pdfpages}

% Sets page size and margins
\usepackage[a4paper, margin=1in]{geometry}

%Line Spacing
\setstretch{1.5}

%-------------------Begin Editing Here---------------------
%Info for Title Page
\title{Z-Transform Operations}
\author{Cameron Williams\\ECE 351-51\\Lab Report 11}
\date{21 April 2020}
\begin{document}

%Make a Title Page
\vspace{\fill}
\maketitle
\vspace{\fill}
\clearpage

\newpage
%Introduction
\section{Introduction}
    \par The objective of this lab was to practice analyzing discrete systems using Python's built-in functions as well as the Z-plane function developed by Christopher Felton.

%Methodology
\section{Methodology}
For the purposes of this lab, the following causal function was provided:
    $$ y[k] = 2x[k] - 40x[k-1] + 10y[k-1] - 16y[k-2] $$
The first task specified was to find the transfer function, H(z):
    \begin{align*}
        y[k] - 10[k-1] + 16y[k-2] &= 2x[k] - 40x[k-1] \\
        Y(z) - 10z^{-1}Y(z) + 16z^{-2} &= 2X(z) - 40z^{-1}X(z) \\
        Y(z)(1 - 10z^{-1} + 16z^{-2} &= X(z)(2 - 40z^{-1} \\
        \frac{Y(z)}{X(z)} &= \frac{2 - 40z^{-1}}{(1-10z^{-1}+16z^{-2})} \\
        H(z) &=\frac{2z(z-20)}{z^2 - 10z + 16} \\
    \end{align*}
    
The second task was to use the derived transfer function to calculate the impulse response via partial fraction expansion:
    \begin{align*}
        \frac{H(z)}{z} &= \frac{2(z-2)}{z^2 - 10z +16} \\
        \frac{H(z)}{z} &= \frac{A}{z-8} + \frac{B}{z-2} \\
        A &= \frac{2(z-20)}{z-8}|_{z=2} = 6 \\
        B &= \frac{2(z-20)}{z-2}|_{z=8} = -4 \\
        \frac{H(z)}{z} &= \frac{6}{z-8} - \frac{4}{z-2} \\
        Z^{-1}\{\frac{H(z)}{z}\} = h[k] &= [6(-8)^k - 4(-2)^k]u[k] \\
    \end{align*}
    
For the next task, I verified the results of my partial fraction expansion using the scipy.signal.residuez() function. The results were as follows:

\begin{verbatim}
Task 3 Residue Results:
 r = [ 6. -4.]
 p = [2. 8.]
 k = []
\end{verbatim}

For Task 4, I passed the transfer function into the provided zplane formula to generate a pole-zero plot. This plot may be seen in Figure 1 of the Results section. Finally, for Task 5, I used the scipy.signal.freqz() function to generate a bode plot for the transfer function H(z). This plot may be seen in Figure 2 of the Results section. Note that this plot wasn't quite right. I worked with the TA for about twenty minutes after the regular lab time trying to resolve this, but we were unable to determine why my plot wasn't being generated correctly. Our code appeared to be functionally identical.

\newpage
%Results
\section*{Results}

\begin{figure}[H]
\centering
\includegraphics[scale=0.45]{include/Figure_1.png}
\caption{Task 4 - Pole-zero plot generated using the provided zplane function.}
\end{figure}

\begin{figure}[H]
\centering
\includegraphics[scale=0.45]{include/Figure_2.png}
\caption{Task 5 - Bode plot showing magnitude and phase response of the transfer function H(z). As discussed in the Methodology section, this plot didn't quite turn out right.}
\end{figure}

\newpage
%Questions
\section*{Questions}

\begin{enumerate}
    \item Looking at the plot generated in Task 4, is H(z) stable? Explain why or why not.
\end{enumerate}
    \par H(z) is not stable as not all of the zeroes and poles are inside of the unit circle projected onto the plot.
    
\begin{enumerate}[resume]
    \item Leave any feedback on the clarity/usefulness of the purpose, deliverables, and expectations for this lab.
\end{enumerate}
    \par The expectations for this lab were expressed clearly in the lab handout.

\end{document}