%%%%%%%%%%%%%%%%%%%%%%%%%%%%%
% Cameron Williams          %
% ECE 351-51                %
% Lab Number 3              %
% 18 February 2020          %
% Instructor Philip Hagen   %
%%%%%%%%%%%%%%%%%%%%%%%%%%%%%

\documentclass[12pt]{article}

% Language and font encoding
\usepackage[english]{babel}
\usepackage[utf8x]{inputenc}
\usepackage[T1]{fontenc}
\usepackage{graphicx}
\usepackage{amsmath}
\usepackage{caption}
\usepackage{float}
\usepackage{caption}
\usepackage{subcaption}
\usepackage{rotating}
\usepackage{setspace}
\usepackage{indentfirst}
\usepackage{enumitem}
\usepackage{appendix}
\usepackage[colorlinks=true, allcolors=blue]{hyperref}
\usepackage{listings}
\usepackage{gensymb}
\usepackage{amssymb}
\usepackage[final]{pdfpages}

% Sets page size and margins
\usepackage[a4paper, margin=1in]{geometry}

%Line Spacing
\setstretch{1.5}

%-------------------Begin Editing Here---------------------
%Info for Title Page
\title{System Step Response Using Convolution}
\author{Cameron Williams\\ECE 351-51\\Lab Report 4}
\date{18 February 2020}
\begin{document}

%Make a Title Page
\vspace{\fill}
\maketitle
\vspace{\fill}
\clearpage

\newpage
%Introduction
\section{Introduction}
    \par The objective of this lab is to become familiar with using convolution to compute a system's step response. To accomplish this, three provided transfer functions will be used in finding their respective step responses via the convolution function created in a previous lab. Additionally, the step responses will be calculated manually and plotted for comparison.
\newpage

%Methodology
\section{Methodology}
    \par I began by importing my convolution, step, and ramp functions created in previous labs. Using the appropriate functions, I created plots of the transfer functions provided. The equations for these transfer functions may be seen below and their plots may be seen in Figure 1 of the results section.
    $$ h_1(t)=e^{2t}u(1-t) $$
    $$ h_2(t)=u(t-2)-u(t-6) $$
    $$ h_3(t)=cos(\omega_0t)u(t) $$
    
    \par Next, I plotted the step response of each of the provided transfer functions. Plots of the three step responses may be seen in Figure 2 of the Results section. I also created plots based on hand calculated step responses to the transfer functions. The calculations for each of the step responses may be seen in the Calculations section and a plot of the results for each may be seen in Figure 3 of the Results section.

\newpage
%Calculations
\section{Calculations}
\begin{align*}
h_1(t)*u(t) &= e^{2t}u(1-t)*u(t) \\
 &= \int_{-\infty}^{\infty} e^{2\tau}u(1-\tau)u(t-\tau)d\tau \\
 &= \int_{-\infty}^{t} u(1-\tau)d\tau + \int_{-\infty}^{t} e^{2\tau}u(\tau - 1)d\tau \\
 &= \int_{\infty}^{t} e^{2\tau}d\tau + \int_{-\infty}^{1} e^{2\tau}d\tau \\
 &= \frac{1}{2}e^{2t}u(1-t)+\frac{1}{2}e^2u(t-1) 
\end{align*}

\begin{align*}
h_2(t)*u(t) &= [u(t-2)-u(t-6)]*u(t) \\
 &= \int_{-\infty}^{\infty} [u(\tau-2)-u(\tau-6)]u(t-\tau)d\tau \\
 &= \int_{-\infty}^{t} u(\tau-2)-u(\tau-6)d\tau \\
 &= r(t-2)-r(t-6)
\end{align*}

\begin{align*}
h_3(t)*u(t) &= [cos(\omega_0t)u(t)]*u(t) \\
 &= \int_{-\infty}^{\infty} [cos(\omega_0\tau)u(\tau)] u(t-\tau)d\tau \\
 &= \int_{-\infty}^{t} cos(\omega_0\tau)u(\tau) d\tau \\
 &= \int_{0}^{t} cos(\omega_0\tau) d\tau \\
 &= \frac{1}{\omega_0}sin(\omega_0t)u(t) \\
 &= \frac{1}{2\pi(.25)}sin(2\pi(.25)t)u(t)
\end{align*}

\newpage
%Results
\section{Results}
    \begin{figure}[hbt!]
        \centering
        \includegraphics[width=\textwidth]{include/Figure_1.png}
        \caption{Plots of transfer functions provided.}
    \end{figure}
    
    \begin{figure}
        \centering
        \includegraphics[width=\textwidth]{include/Figure_2.png}
        \caption{Plots of step responses via my convolution function.}
    \end{figure}
    
    \begin{figure}
        \centering
        \includegraphics[width=\textwidth]{include/Figure_3.png}
        \caption{Plots of calculated step responses.}
    \end{figure}
    
    
\newpage
%Conclusion
\section{Conclusion}
    \par Creating the Python script for this lab was fairly straightforward. However, calculating the convolutions manually gave me some difficulty. I could use more practice with this.
    
\newpage
%Questions
\section*{Questions}

\begin{enumerate}
    \item Leave any feedback on the clarity of lab tasks, expectations, and deliverables.
\end{enumerate}
    \par The lab expectations were communicated clearly.
    
\end{document}